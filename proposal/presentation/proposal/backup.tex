\begin{center}
\begin{figure}[!Htb]
    \begin{tikzpicture} [scale=0.47, transform shape]%show background rectangle,
        \tikzstyle{every node} = [draw, shape = rectangle, node distance=0mm, minimum width=5mm, minimum height=5.2mm]
        \node[draw=blue, thick, label=below:data packet] (dataPacket) {
            \begin{tikzpicture}
                \node (data1) {$1$};
                \node (data2) [right=of data1, xshift=0mm] {$2$};
                \node (data3) [right=of data2, xshift=0mm] {$3$};
            \end{tikzpicture}
        };
        
        \node[draw=black, thick, label=below:Channel 2] (channel2) [right=of dataPacket, xshift=10mm] {
            \begin{tikzpicture}
                \node (puActive) [fill=gray!20, minimum width=30mm] {\small pu active};
                \node (data1) [right=of puActive, xshift=0mm] {$1$};
                %\node (free1) [right=of data1, xshift=0mm] {$2$};
                %\node (free2) [right=of free1, xshift=0mm] {$3$};
                %\node (free3) [right=of free2, xshift=0mm] {};
                %\node (free4) [right=of free3, xshift=0mm] {};
            \end{tikzpicture}
        };
        
        \node[draw=black, thick, label=below:Channel 1] (channel1) [above=of channel2, yshift=5mm] {
            \begin{tikzpicture}
                \node (data1) {$1$};
                \node (data2) [right=of data1, xshift=0mm] {$2$};
                \node (puActive) [right=of data2, xshift=0mm, fill=gray!20, minimum width=25mm] {\small pu active};
                %\node (data3) [right=of puActive, xshift=0mm] {$3$};
            \end{tikzpicture}
        };
        
        \node[draw=red, thick, label=below:Channel 3] (channel3) [below=of channel2, yshift=-5mm] {
            \begin{tikzpicture}
                \node (puActive1) [draw=black, fill=gray!20, minimum width=20mm] {\small pu active};
                \node (data1) [right=of puActive1, xshift=0mm] {$1$};
                \node (data2) [right=of data1, xshift=0mm] {$2$};
                \node (data3) [right=of data2, xshift=0mm] {$3$};
                %\node (free1) [right=of data3, xshift=0mm] {};
                %\node (puActive2) [right=of free1, fill=gray!20, minimum width=15mm] {\small pu active};
            \end{tikzpicture}
        };
        
        \node[draw=blue, thick, right=of channel2, xshift=10mm] (dataPacket2) {
            \begin{tikzpicture}
                \node (data1) {$1$};
                \node (data2) [right=of data1, xshift=0mm] {$2$};
                \node (data3) [right=of data2, xshift=0mm] {$3$};
            \end{tikzpicture}
        };
        %\draw [line width=0.5mm, red] (5.58,-1.54) to (5.58,1.54);
        
        \draw [line width=0.5mm, ->] (1, 0) to (1.75,1.25);
        \draw [line width=0.5mm, ->] (1, 0) -- (1.75, 0);
        \draw [line width=0.5mm, ->, red] (1, 0) to (1.75,-1.25);
        
        
        \draw [line width=0.5mm, ->] (5.8, 1.25) to (6.55,0.1);
        \draw [line width=0.5mm, ->] (5.8, 0) to (6.55,0);
        \draw [line width=0.5mm, ->, red] (5.8, -1.25) to (6.55,-0.1);
    \end{tikzpicture}
        %,height=35mm%64.54mm
        %\caption{Back-up data transmission performs better when primary users remain active or inactive for longer periods}
        \caption{Back-up data transmission}
        \label{fig:motivation-backup}
  \vspace{-0.1in}
\end{figure}
\end{center}
