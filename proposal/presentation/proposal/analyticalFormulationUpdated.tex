{
% Define how TiKZ will draw the nodes
\tikzset{mathterm/.style={draw=white,fill=white,rectangle,anchor=base}}
\tikzstyle{every picture}+=[remember picture]
\everymath{\displaystyle}

\makeatletter

% Designate a term in a math environment to point to
% Syntax: \mathterm[node label]{some math}
\newcommand\mathterm[2][]{%
   \tikz [baseline] { \node [mathterm] (#1) {$#2$}; }}

% A command to draw an arrow from the current position to a labelled math term
% Default color=black, default arrow head=stealth
% Syntax: \indicate[color]{term to point to}[path options]
\newcommand\indicate[2][black]{%
   \tikz [baseline] \node [inner sep=0pt,anchor=base] (i#2) {\vphantom|};
   \@ifnextchar[{\@indicateopts{#1}{#2}}{\@indicatenoopts{#1}{#2}}}
\def\@indicatenoopts#1#2{%
   {\color{#1} \tikz[overlay] \path[line width=1pt,draw=#1,-stealth] (i#2) edge (#2);}}
\def\@indicateopts#1#2[#3]{%
   {\color{#1} \tikz[overlay] \path[line width=1pt,draw=#1,-stealth] (i#2) [#3] edge (#2);}}

\makeatother
\begin{center}
    \begin{align*}
    \visible<1->
    {
    %\begin{gather*}
 \mathterm[t31]{T_{back-up}}&=\min \{T_1,T_2,\ldots, T_x\}\\
    %\end{gather*}
    }
    \visible<3->
    {
    %\begin{gather*}
 \mathterm[t41]{T_{fragmented}}&=\max \{T_1,T_2,\ldots, T_x\}\\
    %\end{gather*}
    }
    \visible<5->
    {
    %\begin{gather*}
       \mathterm[t21]{E[T_{total}]}&=\mathterm[t22]{p_{pu\_idle}}\times\mathterm[t23]{T_{pu\_idle}}\\&+(1-\mathterm[t24]{p_{pu\_idle}})\times(\mathterm[t25]{E[T_{unsuccessful}]}+E[T_{total}])\\
    %\end{gather*}
    }
    \visible<10->{
    %\begin{gather*}
       \mathterm[t1]{T_{pu\_idle}}&=\mathterm[t2]{E[T_{sense}]}+\mathterm[t3]{E[T_{report}]}+\mathterm[t4]{E[T_{negotiate}]}+\mathterm[t5]{T_{data}}\\
    }
    \visible<15->{
    %\begin{gather*}
       E[T_{unsuccessful}] &= \int^{T_{pu\_idle}}_{0}\mathterm[t43]{p_{puc}(t)} t dt\\
       &=\frac{1}{\mathterm[t42]{\lambda_{pu}}}(1-e^{-\lambda_{pu}T_{pu\_idle}}) - T_{pu\_idle}e^{-\lambda_{pu}T_{pu\_idle}}
    %\end{gather*}
    }
    \end{align*}
\end{center}
\only<8>{Total time, if a primary user remains idle\indicate[red]{t23}[out=0,in=-50]}
\only<11>{Channel sensing time\indicate[red]{t2}[out=0,in=-50]}
\only<12>{Reporting time\indicate[red]{t3}[out=0,in=-50]}
\only<13>{Negotiation time\indicate[red]{t4}[out=0,in=-50]}
\only<14>{Data transmission time\indicate[red]{t5}[out=0,in=-50]}
\only<16>{Probability that PU arrives at time $t$\indicate[red]{t43}[out=0,in=-50]}
\only<17>{Primary user arrival rate\indicate[red]{t42}[out=0,in=-50]}
\only<6>{Expected total time \indicate[red]{t21}[out=0,in=-50]}
\only<7>{\small Probability of a primary user of being idle\indicate[red]{t22}[out=50,in=-50]$=e^{-\lambda_{pu}T_{pu\_idle}}$, {\tiny $\lambda_{pu}=$ PU arrival rate}}
\only<9>{Unsuccessful data transmission time if corresponding primary user becomes busy\indicate[red]{t25}[out=0,in=-50]}
\only<2>{Total time for back-up approach\indicate[red]{t31}[out=50,in=-50]}
\only<4>{Total time for fragmented approach\indicate[red]{t41}[out=50,in=-50]}
\only<5>{\begin{center}$T_1,T_2,\ldots, T_x$ are random variables following the same distribution with a mean, $E[T_{total}]$\end{center}}
}
