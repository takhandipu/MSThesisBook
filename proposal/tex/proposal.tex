\documentclass[12pt,addpoints,fleqn]{exam}
%\documentclass[12pt]{article}
\usepackage[legalpaper]{geometry}
\usepackage{array}
\usepackage[T1]{fontenc}
\usepackage{times}
\def\arraystretch{1.5}
\usepackage{booktabs}
\usepackage{textcomp}
\usepackage{censor}
\usepackage{multirow}
\usepackage{amsmath}
\usepackage{graphicx}
\usepackage{enumerate}
\usepackage{soul}
\usepackage{lastpage}
\newcounter{rowcount}
\setcounter{rowcount}{0}
\footer{}{Page \thepage\ of \pageref{LastPage}}{}
%\qformat{\textbf{\thequestion.}}
%\renewcommand{\theenumi}{\Alph{enumi}}
\setul{}{2pt}
\censorruledepth=-.4ex
\censorruleheight=.2ex
\geometry{top=1.0in, bottom=0.8in, left=0.8in, right=0.7in}
\title{test}
\author{test}
\begin{document}
\begin{center}
    {
        \textbf{BANGLADESH UNIVERSITY OF ENGINEERING AND TECHNOLOGY, DHAKA}\\
        \textbf{OFFICE OF THE MEMBER SECRETARY OF THE COMMITTEE FOR\\ADVANCED STUDIES \& RESEARCH, BUET, DHAKA}\\
        \vspace*{0.2cm}
        \textbf{\ul{(Thesis Proposal)}}
    }\\
\end{center}
\begin{minipage}[t]{0.7\textwidth}
\hfill
\end{minipage}
\begin{minipage}[t]{0.3\textwidth}
\flushleft
\textbf{Date:} \today
\end{minipage}
\begin{questions}
{\bfseries \question}
\begin{minipage}[t]{0.7\textwidth}
\flushleft
\textbf{Name of the student:} TANVIR AHMED KHAN\\
\textbf{Roll No.:} 1014052013
\end{minipage}
\begin{minipage}[t]{0.3\textwidth}
\flushleft
\textbf{Status:} Part-time\\
\textbf{Session:} October, 2014
\end{minipage}
{\bfseries \question}
\begin{minipage}[t]{0.7\textwidth}
\flushleft
\textbf{Present Address:} Room No. 420, Dept. of CSE, BUET, DHAKA.
\end{minipage}
\begin{minipage}[t]{0.3\textwidth}
\flushleft
\textbf{Cell No.:} +8801912090363
\end{minipage}
{\bfseries \question}
\begin{minipage}[t]{0.7\textwidth}
\flushleft
\textbf{Name of the Department:} Computer Science and Engineering
\end{minipage}
\begin{minipage}[t]{0.3\textwidth}
\flushleft
\textbf{Programme:} M.Sc. Engineering
\end{minipage}
{\bfseries \question}
\begin{minipage}[t]{0.7\textwidth}
\flushleft
\textbf{Name of the Supervisor:} A. B. M. Alim Al Islam\\
\textbf{Cell No.:} +8801817533953
\end{minipage}
\begin{minipage}[t]{0.3\textwidth}
\flushleft
\textbf{Designation:}\\Assistant Professor, \\Dept. of CSE, BUET.
\end{minipage}
{\bfseries \question}
\begin{minipage}[t]{0.7\textwidth}
\flushleft
\textbf{Name of the Co- Supervisor:} Not Applicable
\end{minipage}
\begin{minipage}[t]{0.3\textwidth}
\flushleft
\textbf{Designation:}\\Not Applicable
\end{minipage}
{\bfseries \question
\textbf{Date of First Enrolment in the Programme:}} Novermber 1, 2014
{\bfseries \question
\textbf{Tentative Title:}} \MakeUppercase{Overcoming Throughput Degradation in Multi-Radio Cognitive Radio Networks.}
{\bfseries \question
\textbf{Background and present state of the problem:}}\\
The famous spectrum scarcity problem along with significant spectrum under-utilization in traditional spectrum management has lead towards the notion of opportunistically dynamic spectrum access~\cite{akyildiz2006next} through cognitive radios~\cite{Mitola}. Cognitive Radio Networks (CRNs) generally exploit the capabilities of cognitive radios in presence of two types of users - Primary Users (PUs), i.e., licensed users, and Secondary Users (SUs), i.e., the unlicensed users employing cognitive radios~\cite{pelechrinis2013cognitive, zhang2016cognitive}.
% A cognitive radio monitors its operational electromagnetic environment to dynamically adjust its operational parameters. Thus, a cognitive radio is capable of accessing temporal free spectrums. Cognitive Radio Networks (CRNs) exploit this capability of cognitive radios to form network connectivities~\cite{zhang2016cognitive}.
%On the other hand, classical wireless networks frequently adopt the notion of deploying users with multiple radios~\cite{bahl2004reconsidering, adya2004multi}. Such deployment of multiple radios enable an increase in bandwidth, leading to an improvement in performance. Therefore, the concept of exploiting multiple radios in CRNs to supplement the dynamic spectrum access has been proposed in the contemporary literature. Such multiradio deployment on CRNs improves delay upto a certain point, however, throughput always degrades with an increase in number of radios per secondary user~\cite{khan2015towards}. To improve network throughput in multi-radio cognitive radio networks, we propose feedback based multi-radio exploitation approach.
On the other hand, classical wireless networks frequently adopt the notion of deploying multiple radios~\cite{bahl2004reconsidering, adya2004multi} to improve performance of the networks~\cite{bahl2004reconsidering, draves2004routing, miu2005improving, song2012performance}. Consequently, it is intuitive that simultaneous utilization of both these techniques, i.e., multi-radio CRNs, will result in significantly improved network performance. Such multi-radio deployment on CRNs improves end-to-end delay up to a certain point, however, throughput degrades with an increase in the number of radios per secondary user~\cite{khan2015towards}. Therefore, the main motivation behind the study is to investigate how to overcome the already-known phenomena of getting degraded network throughput while equipping secondary users with multiple radios.% At the same time, we also want to make sure that the delay metric does not get worse.% while attempting to increase the network throughput.
{\bfseries \question
\textbf{Objectives with specific aims and possible outcome:}}\\
%In this study, we have proposed a feedback based multi-radio exploitation approach for CRNs to improve network throughput. Our proposed approach consists of three steps.
The main objectives of the study are as follows:
\begin{enumerate}[i.]
    \item The first objective of the study is to propose a new approach for multi-radio CRNs that can improve network throughput through exploitation of the multiple radios.
    \item The second objective is to evaluate performance of the proposed approach through experimentation.
    \item The third objective is to compare performance of the proposed approach with that of other contemporary approaches.
\end{enumerate}

The possible outcomes of the study are as follows:
\begin{enumerate}[i.]
    \item A feedback-based multi-radio exploitation approach for CRNs where information obtained from lower layers (Physical layer and Data Link layer) will be incorporated in the process of decision making in an upper layer (Application layer).
    \item An evaluation of the performance of the proposed approach using discrete event simulation through CRE-NS3~\cite{al2014simulating}.
    \item A comparison of performance of the proposed approach against that of the existing approaches available in the literatures for multi-radio CRNs.
\end{enumerate}
\newpage
{\bfseries \question
\textbf{Outline of Methodology/ Experimental Design:}}\\
Outline of the proposed methodology and experimentation can be summarized in the following steps:
\begin{enumerate}
    \item First, a feedback-based approach for multi-radio CRNs will be proposed. In the proposed approach, packet delivery ratio for each radio will be measured to evaluate their individual performance, channel utilization ratio for each channel will be calculated, and finally, the data rate for each radio will be set based on these measurements to maximize the throughput.
    \item Then, modifications in CRE-NS3 simulator required for implementation of the proposed approach will be performed as current version of the simulator does not support the model.
    \item Next, multi-radio CRNs using the developed module will be simulated and the performance of the network will be investigated. Here, operational parameters of the proposed approach will be varied and sensitivity of changing the parameters' values will be evaluated.
    \item Then, the following performance metrics obtained through the proposed approach will be compared against that of the existing approaches.
    \begin{enumerate}[i.]
        \item Average network throughput,
        \item Per node average throughput,
        \item Average end-to-end delay,
        \item Average packet loss.
    \end{enumerate}
    \item Finally, various properties of the proposed approach will be investigated, findings of the study will be discussed, and open issues of the study will be highlighted as future directions.
\end{enumerate}
{\bfseries \question \textbf{References:}}
\vspace{-1.75cm}
\renewcommand\refname{}
\bibliographystyle{ieeetr}%ACM-Reference-Format-Journals
%\bibliographystyle{ACM-Reference-Format-Journals}
%\bibliographystyle{plain}
%\bibliography{ieeetr}
{%\footnotesize
\bibliography{SystemModel}
}
\newpage
{\bfseries \question
\textbf{List of courses taken:}}

 \begin{center}%>{\columncolor[gray]{0.8}}
 \begin{tabular}[c]{|>{CSE 6}p{0.1\textwidth}|p{0.3\textwidth}|>{3.0}p{0.1\textwidth}|>{A+}p{0.1\textwidth}|>{4.0}p{0.1\textwidth}|p{0.1\textwidth}|}
    \hline
    \multicolumn{1}{|c|}{\textbf{Course No}} & \multicolumn{1}{c|}{\textbf{Course Name}} & \multicolumn{1}{c|}{\textbf{Credit}} & \multicolumn{1}{c|}{\textbf{Grade}} & \multicolumn{1}{c|}{\textbf{Grade Point}} & \multicolumn{1}{c|}{\textbf{G.P.A}}\\
    \hline
    806 & Wireless and Mobile Communication Networks & & & & \multirow{7}{0.1\textwidth}{3.75}\\\cline{1-5}
    813 & Network Security & & & & \\\cline{1-5}
    402 & Graph Theory & & \multicolumn{1}{l|}{B} & \multicolumn{1}{l|}{2.5} & \\\cline{1-5}
    602 & High Dimensional Data Management & & & & \\\cline{1-5}
    506 & Data Mining & & & & \\\cline{1-5}
    811 & Wireless Ad Hoc Networks & & & & \\\cline{1-5}
    000 & Thesis & \multicolumn{1}{l|}{18.0} & \multicolumn{1}{l|}{--} & \multicolumn{1}{l|}{--} & \\
    \hline
  \end{tabular}
  \end{center}
\vspace{1cm}
\begin{minipage}[t]{1.0\textwidth}
\flushright
\textbf{Signature of the Tabulator:} \_\_\_\_\_\_\_\_\_\_\_\_\_\_\_\_\_\_
\end{minipage}
{\bfseries \question
\textbf{Cost Estimate:}}
\begin{enumerate}[(a)]
\item Cost of Materials:
\begin{enumerate}[a.]
\item Ink Cartridge: Tk.: 4000/=
\end{enumerate}
\textbf{Total:} Tk.: 4000/=
\item Field works: Not applicable.
\item Conveyance/Data Collection: Not applicable.
\item Research finding outreach: Tk.: 50,000/=
\item Typing, Drafting, Binding, \& Paper etc.:
\begin{enumerate}[a.]
\item Drafting: Tk.: 1250/=
\item Binding: Tk.: 1250/=
\item Paper: Tk.: 1500/=
\end{enumerate}
\textbf{Total:} Tk.: 4000/=
\end{enumerate}
\textbf{Grand Total:} Tk.: 58,000/=
{\bfseries \question
\textbf{Approximate time (in hour) for BUET workshop facilities (if required):}} Not applicable
{\bfseries \question
\textbf{Justification of having Co-Supervisor:}} Not applicable
{\bfseries \question
\textbf{Doctoral Committee/BPGS/RAC reference:}}\\
\begin{minipage}[t]{0.3\textwidth}
\textbf{Meeting No.: } BPGS-2016/10
\end{minipage}
\begin{minipage}[t]{0.3\textwidth}
\textbf{Resolution No.: } 4
\end{minipage}
\begin{minipage}[t]{0.3\textwidth}
\textbf{Date:} July 26, 2016.
\end{minipage}
\iffalse
\begin{minipage}[t]{0.3\textwidth}
\textbf{}
\end{minipage}
\fi
{\bfseries \question
\textbf{Time Extension(if any) up to:}} Not applicable\\
\begin{minipage}[t]{0.3\textwidth}
\textbf{Approved by the CASR\\Meeting No.: }Not applicable
\end{minipage}
\begin{minipage}[t]{0.3\textwidth}
\textbf{Resolution No.: }Not applicable
\end{minipage}
\begin{minipage}[t]{0.3\textwidth}
\textbf{Date:} Not applicable
\end{minipage}
{\bfseries \question
\textbf{Appointment of Supervisor \& Co-Supervisor Approved by the CASR Meeting No. (For Ph. D):}} Not applicable\\
\begin{minipage}[t]{0.45\textwidth}
\textbf{Resolution No.: }Not applicable
\end{minipage}
\begin{minipage}[t]{0.45\textwidth}
\textbf{Date:} Not applicable
\end{minipage}
{\bfseries \question
\textbf{Appointment of Doctoral Committee Approved by the CASR Meeting No. (For Ph. D):}} Not applicable\\
\begin{minipage}[t]{0.45\textwidth}
\textbf{Resolution No.: }Not applicable
\end{minipage}
\begin{minipage}[t]{0.45\textwidth}
\textbf{Date:} Not applicable
\end{minipage}
{\bfseries \question
\textbf{Result of the comprehensive examination for Ph. D.:}} Not applicable
{\bfseries \question
\textbf{Number of Post-Graduate Students working with the Supervisor at Present:}}
M.Sc. (21), Ph.D. (3)\\
\begin{minipage}[t]{0.4\textwidth}
\flushleft
\vspace{3cm}
\_\_\_\_\_\_\_\_\_\_\_\_\_\_\_\_\_\_\_\_\_\_\_\_\_\_\_\\
Signature of the Student\\
\vspace{3cm}
\_\_\_\_\_\_\_\_\_\_\_\_\_\_\_\_\_\_\_\_\_\_\_\_\_\_\_\\
Signature of the Supervisor\\
\vspace{3cm}
\_\_\_\_\_\_\_\_\_\_\_\_\_\_\_\_\_\_\_\_\_\_\_\_\_\_\_\\
Signature of the Head/Director\\
\end{minipage}
\begin{minipage}[t]{0.5\textwidth}
\flushleft\vspace{1cm}
{\begin{center} Names and signatures of the members of the Doctoral Committee (if applicable) \end{center}}
\begin{center}
\def\arraystretch{2.25}
\begin{tabular}{|@{\stepcounter{rowcount}\makebox[2em][c]{\therowcount.}}|p{0.8\textwidth}|}
\hline
\\
\hline
\\
\hline
\\
\hline
\\
\hline
\\
\hline
\\
\hline
\\
\hline
\\
\hline
\end{tabular}
\end{center}
\end{minipage}
\end{questions}
\end{document}
