\chapter{Conclusion and Future Work}
Cognitive radio networks suffer noteworthy throughput degradation with the introduction of multi-radio usage. We propose a feedback-based multi-radio exploitation approach for CRNs in this thesis to overcome this throughput degradation. We implement the proposed approach in ns-3 to measure various performance metrics such as throughput, delay, packet delivery ratio, and packet drop ratio over several network settings. Simulation results reveal that our proposed approach can significantly increase total network throughput along with decreasing packet drop ratio and delay compared to other existing techniques. Furthermore, the feedback-based approach can be used to find a suitable number of radios needed to experience a delicate tradeoff between network throughput and delay for applications maintaining different data rates. In future, we plan to formulate analytical models of our proposed approach and implement the approach in real testbed.

In our study, we perform extensive simulations to validate our proposed approach. In future, we plan to validate our presented solution over CR testbed. We also plan to formulate analytical models of the solution in future. Here, our target is to model the probability of successful packet transmission and the probability of selecting PU-free channels. From these two models, we plan to formulate separate models for delay and throughput of our proposed MRCRN architecture. Besides, multi-path communication via multiple cognitive radios would be another interesting field to study. In future, we plan to investigate the performance of MRCRNs exploiting multi-path communication.
\endinput
