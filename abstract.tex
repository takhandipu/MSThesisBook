\chapter*{Abstract}
\chaptermark{Abstract}
%\pagenumbering{roman}
%\newpage
%\begin{center}  
%	{\Huge \textbf{ Abstract}}
%\end{center}
\addcontentsline{toc}{chapter}{\textit{Abstract}} In recent years, Cognitive Radio Networks (CRNs) have been widely investigated to solve the well-known spectrum scarcity problem through enhancing spectrum utilization. Another technique of enhancing spectrum utilization, which has already been well accepted, is to utilize multiple radios on a single node. Simultaneous usage of both these techniques is therefore expected to enhance the spectrum utilization further in road to improving overall network performance. However, little research efforts have been spent on investigating performance of the simultaneous usage through incorporating multiple radios in each node of a CRN. Existing studies in this regard propose several protocols for Multi-Radio Cognitive Radio Networks (MRCRNs). However, none of them focuses on increasing throughput in the network to the best of our knowledge. Nonetheless, increased network throughput should be a direct consequence of enhanced spectrum utilization through exploiting multiple radios in CRNs, even though an existing literature~\cite{khan2015towards} reports getting decreased network throughput while introducing multiple radios in CRNs. Thus, a specialized treatment to multiple radios in CRNs is needed for increasing network throughput. Accordingly, in this study, we propose a feedback-based multi-radio exploitation approach for MRCRNs, where information obtained from lower layers (Physical layer and Data Link layer) is incorporated in the process of decision making in an upper layer (Application layer) to enhance network throughput. We implement our proposed approach in \texttt{ns-3} to measure different performance metrics including network throughput, average end-to-end delay, and average packet drop ratio. We compare the performance against that of existing multi-radio exploitation approaches for CRNs. Our simulation results reveal that our proposed feedback-based approach always achieves substantially increased network throughput compared to existing approaches, in parallel to achieving improved delay and packet drop-ratio in most of the cases.
