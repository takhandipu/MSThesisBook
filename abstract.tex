\chapter*{Abstract}
\chaptermark{Abstract}
%\pagenumbering{roman}
%\newpage
%\begin{center}  
%	{\Huge \textbf{ Abstract}}
%\end{center}
\addcontentsline{toc}{chapter}{\textit{Abstract}} In recent years, Cognitive Radio Networks (CRNs) have been widely investigated to solve the spectrum scarcity problem. Another well-accepted technique to enhance network performance is to utilize multiple radios on a single node. Simultaneous usage of both these techniques is therefore expected to improve network performance. However, little research efforts have been spent in incorporating multiple radios for CRNs. Existing studies propose several medium access control (MAC) and routing protocols for Cognitive Multi-Radio Networks (CMRNs), however, none of them focuses on enhancing throughput in the network to the best of our knowledge. Therefore, in this study, we propose a feedback-based multi-radio exploitation approach for CRNs where information obtained from lower layers (Physical layer and Data Link layer) is incorporated in the process of decision making in an upper layer (Application layer) to enhance network throughput. We implement our proposed approach in \texttt{ns-3} to measure different performance metrics including throughput, average end-to-end delay, and average packet drop ratio. We compare the performance against that of existing multi-radio approaches. Our simulation results suggest that the proposed feedback-based approach always achieves substantially improved network throughput compared to existing approaches, in parallel to achieving improved delay and packet drop-ratio in most of the cases.
